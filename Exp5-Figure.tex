\documentclass[border=15mm]{standalone}
\usepackage{circuitikz}
\usepackage{relsize}
\usepackage{newtxtext}
\usepackage{newtxmath}
\usepackage[utf8]{inputenc}
\usetikzlibrary{calc} % برای استفاده از coordinate محاسباتی ($ ... $)

\begin{document}

\makebox[\textwidth][c]{
\begin{circuitikz}[scale=1.4]

% نودها
\coordinate (1) at (-5.17, -3.91);
\coordinate (2) at (4.57, -3.89);
\coordinate (3) at (-0.33, -1.8); % تغییر مختصات y نقطه 3 برای افزایش ارتفاع
\coordinate (4) at (-1.95, 0.13);
\coordinate (5) at (1.31, 0.17);
\coordinate (6) at (-1.91, 3.37);
\coordinate (7) at (1.29, 3.41);

% نقاط میانی مسیرها
\coordinate (batCapMid) at ($ (1)!0.47!(2) $);
\coordinate (capResMid) at ($ (1)!0.53!(2) $);
\coordinate (batCapMidB) at ($ (1)!0.47!(6) $);
\coordinate (capResMidB) at ($ (1)!0.53!(6) $);
\coordinate (batCapMidC) at ($ (2)!0.48!(7) $);
\coordinate (capResMidC) at ($ (2)!0.52!(7) $);
\coordinate (mid45) at ($ (4)!0.61!(5) $); % نود میانی نزدیک‌تر به نود 5

% پس‌زمینه سفید
\fill[white] (1) -- (2) -- (3) -- (4) -- (5) -- (6) -- (7) -- cycle;

% مقاومت‌ها و خازن‌ها و منابع — با تغییرات موقعیت برچسب‌ها
\draw (2) to[R, a={\large R}] (3);        % ✅ برچسب بالای مقاومت (بین 2 و 3)
\draw (3) to[R={\large R}] (4);
\draw (4) to[R={\large R}] (mid45);       % مقاومت بین نود 4 و نود میانی
\draw (mid45) to[C, l=C] (5);             % خازن بین نود میانی و نود 5
\draw (5) to[R={\large R}] (7);
\draw (6) to[R={\large R}] (7);
\draw (6) to[R={\large R}] (4);
\draw (2) to[R={\large R}] (5);
\draw (1) to[R={\large R}] (3);

\draw (1) to[battery2, l={\Large $e_1$}] (batCapMid);
\draw (batCapMid) to[C, l=C] (capResMid);
\draw (capResMid) to[R={\large R}] (2);

\draw (1) to[battery2, l={\Large $e_2$},invert] (batCapMidB);
\draw (batCapMidB) to[C, l=C] (capResMidB);
\draw (capResMidB) to[R={\large R}] (6);

\draw (2) to[battery2, name=bat27] (batCapMidC);
\draw (batCapMidC) to[C, name=cap27] (capResMidC);
\draw (capResMidC) to[R, l_={\large R}] (7); % ✅ برچسب سمت راست مقاومت (بین 2 و 7)

\node[above=10pt] at (bat27) {\Large $e_3$};
\node[above=11pt] at (cap27) {C};

% نودها به صورت دایره سفید با شماره آبی
\foreach \point in {1,2,3,4,5,6,7}
    \node[circle, draw=blue, fill=white, text=blue,
          minimum size=4mm, inner sep=0pt,
          font=\bfseries\footnotesize] at (\point) {\point};

\end{circuitikz}
}

\end{document}